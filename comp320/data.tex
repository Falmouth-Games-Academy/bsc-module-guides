\def\moduleCode{COMP320}
\def\moduleName{Research Practice}
\def\credits{20}
\def\isCompulsory{Compulsory}

\def\moduleLeader{Dr Edward Powley}
\def\academicStaff{Dr Michael Scott (Moderator)}

\def\introduction{You are required to deliver a major research project as part of your degree; either in the form of empirical research relating to computing for games, or practice-based research related to game development. Individually, you explore a field that interests you, and for which there is a clearly identified need. This module forms the first part of this project and provides the opportunity to conduct a literature review, as well as to collect and analyse data using appropriate methods and statistics.}

\def\moduleAims{
	{Develop a research question and analyse methods of research appropriate to that question.},
	{Consolidate knowledge and experience of how to organise and execute a non-trivial computing project.},
	{Professional apply research methods in computing.}}
	
\def\moduleLearningOutcomes{1,2,3,4,6}
	
\def\assessmentCriteria{
	{Apply principles of computing creatively to build iteratively an effective computing solution relevant to the development of games.},
	{Communicate in an academic format.},
	{Analyse critically the strengths and weaknesses of your iterations and work iteratively on the basis of on-going evaluation.},
	{Create a solution for which there is a market and for which you can show need.},
	{Create an innovative solution developed for a known target market.},
	{Make use of a range of methods to organise and execute a computing solution, meet deadlines, plan and organise your work flow effectively.}}

\def\assignments{
	{Prototype Research Artefact, 30\%},
	{Research Review \& Proposal, 70\%}}

\def\hours{
	{Sessions, 24 hours},
	{Research Supervision, 4 hour},
	{Directed Reading, 12 hours},
	{Prototype Research Artefact, 20 hours},
	{Integration into Collaborative Game, 20 hours},
	{Research Review \& Proposal, 40 hours},
	{Self-Directed Study, 40 hours},
	{Self-Directed Studio Practice, 40 hours}}
	
\def\learningSpaceID{3029}