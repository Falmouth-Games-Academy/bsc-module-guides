\def\moduleCode{COMP130}
\def\moduleName{Game Architecture \& Engineering}
\def\credits{40}
\def\isCompulsory{Compulsory}

\def\moduleLeader{Dr Michael Scott}
\def\academicStaff{Dr Edward Powley (Moderator), Brian McDonald (Moderator)}

\def\introduction{This module extends your game development practice by getting you to  engage with the principles of professional software engineering in the context of a collaborative multi-disciplinary project. All the while, researching the importance of software quality and applying your findings to shape, measure, and improve the computing solutions that you integrate into your game.}

\def\moduleAims{
	{Acquire knowledge of professional software architecture and engineering in the context of games.},
	{Apply metrics and re-factoring practices to the evolution of a game architecture in a collaborative context.},
	{Implement software design principles and engineering practices at a foundational level.}}
	
\def\moduleLearningOutcomes{1,2,3,4,5,6}
	
\def\assessmentCriteria{
	{Understand the fundamental use of development tools, how games vary between different architectures, and the importance and methods of reuse and scalability within professional software engineering.},
	{Show a basic understanding of how to communicate effectively with stakeholders in writing, and through adherence to coding standards. Annotate software to communicate with others effectively.},
	{Analyse critically the strengths and weaknesses of your code and develop an ability to respond to the critical judgements of others. Identify recurring problems across diverse examples in order to build collective solutions.},
	{Apply basic research methodologies to draw upon existing bodies of knowledge in professional software engineering to understand developments in game architectures, notably design patterns as they occur in games development.},
	{Demonstrate an understanding of the commercial and enterprise constraints that game markets place on technical decisions through requirements to engineer extensible and adaptable solutions.},
	{Show an understanding of how to plan and manage time, meet deadlines by planning available time effectively.}}

\def\assignments{
	{Software Engineering Essay, 30\%},
	{Production Tasks, 40\%},
	{Game Demo, 10\%},
	{CPD Tasks, 20\%}}

\def\hours{
	{Sessions, 48 hours},
	{Supervised Studio Practice, 94 hours},
	{Directed Reading, 36 hours},
	{Software Engineering Essay, 42 hours},
	{Production Tasks, 56 hours},
	{Game Demo Preparation, 14 hours},
	{CPD Tasks, 28 hours},
	{Self-Directed Game Development Practice, 42 hours},
	{Self-Directed Studio Practice, 40 hours}}
	
\def\learningSpaceID{2997}