\chapter{Orientation}
\newpage

\section{Welcome to Falmouth}

First and foremost, a very warm welcome to Falmouth from the \textit{BSc(Hons) Computing for Games} team. We will endeavour to make your time at Falmouth University productive, fun and invigorating. Please don't hesitate to talk to us if you have any questions or issues about your learning experience. \\

Computing technology has changed our lives; not only in the ways we work and communicate, but also the ways in which we play games. This degree will help you to build the skills you need to create and manipulate the computational technologies that will shape the games of the future. \\

The creative industries are booming in the UK. With games reaching across a multitude of platforms and attracting new markets there are now a range of employment opportunities, with plenty of room for independent start-ups. Learning how to use and manage computers and information technology creatively and skilfully will, then, equip you highly-demanded knowledge, skills, and experience; both, within the games industry and further afield. You will learn to apply computing techniques to solve a variety of problems, using a range of programming languages and technologies. You will also learn to take an agile approach to devising computing solutions to game development challenges. Combining elements of creative computing with software engineering, the course leverages the principles of computer science and applies them to the specific challenges and opportunities that digital games present. Along with our emphasis on creativity and enterprise, this ensures that you're able to innovate, develop your own intellectual property and take it to market. With technical competencies in the areas of programming and software development, you'll be able to collaborate with art-based students to develop novel and appealing entertainment software. \\

\section{Meet the Team}

\subsection[Tanya Krzywinska]{Professor Tanya Krzywinska,\\ Director of the Games Academy}

Tanya started playing computer games when she worked at Digital Equipment Company while doing a Masters in film.  After completing her PhD and teaching film and media, she realised that there was very little written about digital games in an academic context and decided to remedy that.  She gave a paper on Resident Evil at one the first academic conferences on games in 2000 and edited with Geoff King the first academic collection devoted to the study of the relationship between games and cinema. Since then she has focused pretty much on games, and is particularly interested in ‘worlds’. She has worked for over twenty years teaching film and games, convening the Games BA, MA and PhD programme at Brunel University. She began working at Falmouth in 2013, after doing a part-time MA in Authorial illustration, tasked with setting up a suite of new games courses and leading research into games. \\

She is the author of several books and numerous articles on film and games including: A Skin for Dancing In: Possession, Witchcraft and Voodoo in Film (Flicks Books, 2000); Sex and the Cinema (Wallflower, 2006); co-author with Geoff King of Tomb Raiders and Space Invaders: Videogames Forms and Contexts (IB Tauris, 2006); and co-editor of ScreenPlay: cinema/videogames /interfaces (Wallflower, 2002); videogame/player/text (MUP, 2006); and Ring Bearers (MUP 2010). She also guest edited an edition of Games and Culture (Sage) devoted to the analysis of World of Warcraft (2006). She has been President of DiGRA(www.digra.org), and is currently the editor of the journal Games and Culture, as well as on the editorial board of journals Slayage, Game Studies, Games and Simulation, GAME, Journal of Virtual Worlds and Horror Studies. \\

\subsection[Simon Colton]{Professor Simon Colton,  Research Chair}

Simon is a leading researcher in the field of Artificial Intelligence, specialising in questions of Computational Creativity. In particular, he develops and investigates novel AI techniques and then applies them to creative tasks in domains such as pure mathematics, graphic design, video game design, creative language, and the visual arts. By taking an overview of creativity in such domains, he hopes to add to the philosophical discussion of creativity by addressing issues raised by the idea of autonomously creative software. 

He is most well known for his work developing The Painting Fool, a computer program that he hopes will be taken seriously as a real artist. More recently, however, he has been investigating the potential for Computational Creativity to enhance video game design, and vice versa. One output of this research is ANGELINA, a program that designs games. He has published over 170 peer-reviewed articles and won the prestigious British Computing Society Machine Intelligence Award in 2007. In May 2014, he was awarded €2.4m to promote research excellence in digital 

\subsection[Nick Dixon]{Nick Dixon, Director of LaunchPad}

Nick Dixon is a games industry veteran with over 15 years' experience in game design, creative direction and pitching. Nick has published 11 games across 12 different platforms and has written the screenplays for 4 of those titles. Nick has created successful concepts for high profile franchises, including Star Wars, Stargate and Doctor Who and has worked with BBC Worldwide, Universal, Laika Studios, MGM and Warner Bros. His game concepts have signed to generate over £30 million of development business for the studios that he has worked with. Nick joins the University to assist in the development of the LaunchPad Falmouth programme and advise on the development of its undergraduate degree courses. 

\subsection[Douglas Brown]{Douglas Brown, Head of Game Courses}

Doug has been a gamer his whole life and has always been fascinated by games of all kinds, particularly their potential for storytelling and reshaping of narrative experiences. After graduating from Oxford, Doug worked in the localisation department of Square-Enix from 2006-2010, and is credited on several of their titles from that period. He eventually joined academia full-time, teaching and then leading the well respected Games Design courses at Brunel University. Doug passionately believes in games as a new field of academic study, and it is his personal mission to see games as respected as other media in higher education. After seeing what Tanya was building in Falmouth, he was convinced of the potential of this new approach to games teaching, and moved down to join the team in 2014. \\

Doug has published numerous book chapters and conference papers on games. He's also been interviewed by the BBC about games. He completed a PhD focused on Games and Suspension of Disbelief in 2012, which is currently being turned into a book. Right now he is also writing on rogue-like games and franchise adaptations. \\

\subsection[Michael Scott]{Dr Michael Scott, \\ Course Coordinator for Computing for Games \& Creative App Development}

Michael leverages the skills and experience he developed reading Computer Science and Digital Games Theory to investigate game procedurality, accessible game interfaces, and game-inspired educational approaches. He is particularly interested in how lusory perspectives can be applied to teaching computer programming, having completed a PhD in this area. Alongside his teaching role on the Computing for Games programme, he can often be found experimenting with multimedia-based instructional technology.\\

Michael was previously an indie developer, working with Emotional Robots during the development of iOS FPS title Warm Gun, before embarking on series of international research projects. His research has been published in ACM Multimedia, IEEE Transactions on Education, Computer Standards \& Interfaces, as well as the New Review of Hypermedia \& Multimedia.\\

\subsection{Dr Edward Powley, Senior Lecturer}

Joining the teaching team alongside his role as a Research Fellow in Computational Creativity at Falmouth's Academy of Innovation and Research, Ed will be lending his considerable technical knowledge and specialist expertise to the course. His areas of interest include procedural content generation, artificial intelligence, and programming. \\

Ed's research has been published across a range of peer-reviewed academic publications including Artificial Intelligence, the IEEE Transactions on Computational Intelligence and AI in Games, and the AAAI Conference of Artificial Intelligence and Interactive Digital Entertainment. \\

\subsection{Brian McDonald, Senior Lecturer}

Joining the teaching team alongside his role as a Research Fellow in Computational Creativity at Falmouth's Academy of Innovation and Research, Ed will be lending his considerable technical knowledge and specialist expertise to the course. His areas of interest include procedural content generation, artificial intelligence, and programming. \\

Ed's research has been published across a range of peer-reviewed academic publications including Artificial Intelligence, the IEEE Transactions on Computational Intelligence and AI in Games, and the AAAI Conference of Artificial Intelligence and Interactive Digital Entertainment. \\

\subsection{Alcwyn Parker, Lecturer}

Joining the teaching team alongside his role as a Research Fellow in Computational Creativity at Falmouth's Academy of Innovation and Research, Ed will be lending his considerable technical knowledge and specialist expertise to the course. His areas of interest include procedural content generation, artificial intelligence, and programming. \\

Ed's research has been published across a range of peer-reviewed academic publications including Artificial Intelligence, the IEEE Transactions on Computational Intelligence and AI in Games, and the AAAI Conference of Artificial Intelligence and Interactive Digital Entertainment. \\

\subsection{Martin Cooke, Lecturer}

Martin entered the world of games programming over 20 years ago and has been hooked ever since. Using languages including assembler, BASIC, C++, Java, and C\#, Martin has developed for numerous different platforms including desktop, web and mobile. Martin is the director of local game makers’ community group and a veteran of game jams including Ludum Dare and Indie Speed Run. Recently Martin worked as a Technical Programmer with Anti-Matter Games, prototyping new concepts and contributing code to the critically-acclaimed title Rising Storm.\\

\subsection{Rich Barham, Senior Lecturer}

Rich brings an unrivalled array of games industry experience to the team as well as connections to some of the biggest names in gaming.  He worked as a games industry executive for over 12 years, with credits including: World of Warcraft, Burning Crusade, Wrath of the Lich King, League of Legends, Elder Scrolls Online and Hitman. Rich served as a director at Zenimax Online Studios, Riot Games, IO-Interactive, and was also a member of the leadership team which built Blizzard Entertainment. He also has experience in games entrepreneurship as the previous CEO of Octopus 8 studios, one of Develop-Online's most exciting 2014 Game Startups.\\

\subsection{Andy Smith, Game Studio Supervisor \& Technician}

A graduate of Falmouth's BA programme in Digital Media, Andy supports the Games Academy as a general technician, technologist and software developer. He manages the IT in the Game Studio and is happy to answer any technical queries that students may have. He is also the Managing Director of Joint Effort Studios, a consultancy in creative technology.\\

\section{Communication}

Your main points of contact on the Computing for Games course will be Dr. Michael Scott (Course Coordinator) and for senior administrative matters Dr. Douglas Brown (Head of Game Courses). You can get in touch with them by using the following email addresses:\\

michael.scott@falmouth.ac.uk \\
douglas.brown@falmouth.ac.uk\\

Academic support and guidance will be offered to you by tutors within each module. However, you will also be supported by a Personal Tutor who will guide your academic progress throughout your studies. Please note, that you will be allocated a Personal Tutor in your first week.\\

A timetable showing games staff’ office hours will be posted on the Learning Space, the Virtual Learning Environment used by Falmouth University. You are welcome to make appointments to see the team during these hours. To do this, use Scheduler Tool on the Learning Space. The Learning Space also acts as a central location for you to find up to date information on specific modules you are taking, timetables and room assignments, handbooks, assignments, reading materials, staff appointments, and so on. As such, you will want to become familiar with this site and use it daily.\\

Please note, that periodically the teaching team will want to contact you. This will be done using your official Falmouth email account, so please check this regularly. \\

Also note that you may want to engage with fellow staff and students through the FaceBook Group and the IRC tool.\\

\chapter{Studio Practice}
\newpage

3.1.	Expectation of Studio Attendance
During collaborative projects, you are expected to be in the studio working on your games during normal office hours (9am - 5pm). Working together making use of the facilities encourages teamwork and communication skill development. Technical help is available from the Studio Technician.  The studio is open until midnight on working days, and 9-5 on weekends.

3.2.	Use of the Games Teaching Space
The Games Teaching Space is available for use outside of scheduled classes. Note, however, that the space is used to teach other students and so, when required, you will be expected to vacate the room in a quick and professional manner.

3.3.	Professionalism

All persons using the studio must:

Be courteous towards all students, staff and visitors making use of the studio;
Take responsibility for maintaining a high degree of cleanliness of the studio;
Dispose of all discarded items in the recycling / litter bins provided;
No food is to be consumed in the studio;
All drinks must be kept in spill-proof containers;
Keep noise to a reasonable minimum and where necessary use headphones;
Mobile phones must be on low or silent and any calls must be taken outside of the studio.

3.4.	Health \& Safety
All persons using the studio must:

Know the location of the nearest emergency telephone point and the studio first aid kit;
Maintain an up to date knowledge of fire safety procedures including alarm sound, exit routes and assembly points;
Immediately report occurrences of any person within the studio falling ill, being involved in an accident or otherwise sustaining any injury;
Immediately report any hazards observed within the studio, including:
Damaged electrical equipment;
Water leaks;
Broken fixtures;
Trip hazards;
Maintain an up to date awareness of health and safety practices relating to studio workstations and display screen equipment;
Ensure all bags and personal belongings are stored away from walkways and fire exits.

3.5.	Unacceptable Use of Computers
All persons using the studio must not…

Create, copy or display any offensive, obscene or indecent material;
Create, copy or display any defamatory material;
Engage in plagiarism or infringe upon copyrights;
Attempt to gain unauthorised access to any restricted facilities (physical or digital);
Deliberately introduce any form of malware or viruses to the studio computers.

3.6.	Security
All persons using the studio must not…

Invite third parties into the studio;
Permit entry to unauthorised persons;
Wedge fire and/or security doors open.
