\moduleCode{COMP120}
\moduleName{Creative Computing: Tinkering}
\credits{20}
\isCompulsory{Compulsory}

\moduleLeader{Brian McDonald}
\def\academicStaff{Dr Michael Scott (Moderator)}

\def\introduction{This module is designed to help you learn different ways of engaging with code using practical and explorative methods. You will learn the value of taking a creative approach to computing, taking existing code and modifying it in creative ways. The module will introduce you to the core principal of rapid iteration, where tinkering with existing code can provide the basics on which something new can be built.}

\def\moduleAims{
	{Understand computing for games},
	{Understand how to re-purpose and augment code to build something new},
	{Apply programming skills creatively}}
	
\def\moduleLearningOutcomes{1,2,3,5,6}
	
\def\assessmentCriteria{
	{Focusing on software engineering, show ability to modify and repurpose existing code and create demonstrations of digital programming in response to briefs.},
	{Annotate software clearly, articulate clearly, and succinctly your evaluation of your working practice.},
	{Evaluate your working practice showing that you understand the analytical approach required to learn from your practical work.},
	{},
	{Show ability to creatively repurpose existing code appropriately and understand the fundamentals of a creative approach to computing.},
	{Show understanding of agile methods and meet deadlines by planning available time effectively.}}

\def\assignments{
	{Code Repurposing I --- Tinkering Graphics, 30\%},
	{Code Repurposing II --- Tinkering Audio, 70\%}}

\def\hours{
	{Sessions, 36 hours},
	{Directed Reading, 18 hours},
	{Graphics Programming, 21 hours},
	{Audio Programming, 49 hours},
	{Self-Directed Programming Practice, 36 hours},
	{Self-Directed Studio Practice, 40 hours}}
	
\def\learningSpaceID{1250}