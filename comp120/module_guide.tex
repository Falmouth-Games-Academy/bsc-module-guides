\documentclass[handout, xcolor={dvipsnames}]{beamer}\usepackage{etoolbox}\newtoggle{printable}\toggletrue{printable}

% Adjust these for the path of the theme and its graphics, relative to this file
\usepackage{../beamerthemeFalmouthGamesAcademy}
\usepackage{multimedia}
\graphicspath{ {../} }

% Additional symbols
\usepackage{wasysym}

\begin{document}

\title{COMP120}
\subtitle{Creative Computing: Tinkering }
\moduleCode{COMP120}
\moduleLeader{Brian McDonald}
\credits{20}
\isCompulsory{Compulsory}

\frame{\titlepage} 

\begin{frame}
	\frametitle{Introduction}
This module is designed to help you learn different ways of engaging with code using practical and explorative methods. You will learn the value of taking a creative approach to computing and become acquainted with some of the principles behind Creative Computing. You will take existing code and modify it in creative ways in order to build your understanding of computing for games and to create repurposed use of code and technology. The module will introduce you to the core principal of rapid iteration, where tinkering with existing code can provide the basics on which something new can be built. You will therefore gain experience of how to produce innovation and efficiency through the principles of repurposing. You work with open source material and gain a practical understanding of the value and use of open source in a creative computing and games context. 
	
\end{frame}

\begin{frame}
	\frametitle{Aims}
	\begin{itemize}
		\item Understand computing for games
		\item Understand how to re-purpose and augment code to build something new
		\item Apply programming skills creatively
	\end{itemize}
\end{frame}

\begin{frame}
	\frametitle{Overview}
	
	Grid goes here.
\end{frame}

\begin{frame}
\frametitle{Learning Outcomes}

\centering
\tiny
\begin{tabular} { | p{.05\textwidth} | p{.4\textwidth} | p{.4\textwidth} |}
	%HEAD
	LO &
	\textbf{Learning Outcomes} Upon completion of this Module you should be able to: &  
	\textbf{Assessment Criteria} To achieve the learning outcome you must demonstrate the ability to: \\
	\hline
	% OUTCOME 1
	1 &
	Show a basic understanding of creative computing solutions using professional techniques  &  
	Demonstrate a basic understanding of computing fundamentals. Apply basic knowledge and understanding of the techniques used in software development. Understand the creative value of maker-style and iterative approaches for the generation of innovation. \\
	% OUTCOME 2
	2 &
	Show a basic understanding of how to communicate effectively with stakeholders in writing, verbally and through adherence to coding standards &  
	Show a basic understanding of how to communicate effectively with stakeholders in writing, verbally, and through adherence to coding standards. Annotate software to communicate with others effectively. \\
	% OUTCOME 3
	3 &
	Show a basic development of the ability to reflect critically on and evaluate working methods and solutions  & 
	Analyse critically the strengths and weaknesses of code and develop an ability to respond to the critical judgements of others. \\
	% OUTCOME 4
	4 &
	Show a basic understanding of the ability to conduct research, present knowledge in an academic format and apply that research to practice  & 
	Research and explain the use of methodologies used in computing, apply knowledge to practice, and present that knowledge where appropriate in an academic format. \\
	% OUTCOME 6
	6 &
	Show a basic understanding of methods used to help set goals, manage workloads to meet deadlines and to work collaboratively  & 
	Set goals and manage workloads to meet deadlines using set methodologies and present ideas in a variety of situations with appropriate support. 
\end{tabular}


\end{frame}

\begin{frame}
\frametitle{Additional Resources}

Session Plans \& Materials: \\
\url{http://learningspace.falmouth.ac.uk/}

\vspace{1.5em}	

Assignment Briefs: \\
\url{https://github.com/Falmouth-Games-Academy/bsc-assignment-briefs}

\vspace{1.5em}	

Reading List: \\
\url{http://resourcelists.falmouth.ac.uk/index.html}

\end{frame}

\end{document}