\documentclass[handout, xcolor={dvipsnames}]{beamer}\usepackage{etoolbox}\newtoggle{printable}\toggletrue{printable}

% Adjust these for the path of the theme and its graphics, relative to this file
\usepackage{../beamerthemeFalmouthGamesAcademy}
\usepackage{multimedia}
\graphicspath{ {../} }

% Additional symbols
\usepackage{wasysym}

\begin{document}

\title{COMP110}
\subtitle{Principles of Computing}
\def\moduleLeader{Dr Edward Powley}
\def\academicStaff{Dr Michael Scott}
\credits{20}
\isCompulsory{Compulsory}

\frame{\titlepage} 

\begin{frame}
	\frametitle{Introduction}
	This module is designed to introduce you to the basic principles of computing in the context of digital games. It is designed to complement the other modules, providing a broad foundation on the theories, methods, models, and techniques in computing which will help you to construct computer programs and be able to make use of relevant scholarly sources.
	
\end{frame}

\begin{frame}
	\frametitle{Aims}
	
	This module aims to help you:
	
	\begin{itemize}
		\item Understand the basic principles, terminology, roles, and software development concept that computing professionals apply within a game development context 
		\item Understand how to apply computing theory to practical programming activities
		\item Understand how to conduct basic software development tasks
	\end{itemize}
\end{frame}

\begin{frame}
	\frametitle{Learning Outcomes}
	
	\centering
		\tiny
		\def\arraystretch{1.5}
		\begin{tabular} { | p{.02\textwidth} | p{.45\textwidth} p{.45\textwidth} |}
			%HEAD
 			\textbf{LO} &
			\textbf{Learning Outcomes} &  
			\textbf{Assessment Criteria} \\

			% OUTCOME 1
 			1 &
			Show a basic understanding of creative computing solutions using professional techniques  &  
			Demonstrate a basic understanding of computing fundamentals. Apply basic knowledge and understanding of the techniques used in software development. Understand the creative value of maker-style and iterative approaches for the generation of innovation. \\
			% OUTCOME 2
 			2 &
			Show a basic understanding of how to communicate effectively with stakeholders in writing, verbally and through adherence to coding standards &  
			Show a basic understanding of how to communicate effectively with stakeholders in writing, verbally, and through adherence to coding standards. Annotate software to communicate with others effectively. \\
			% OUTCOME 3
 			3 &
			Show a basic development of the ability to reflect critically on and evaluate working methods and solutions  & 
			Analyse critically the strengths and weaknesses of code and develop an ability to respond to the critical judgements of others. \\
			% OUTCOME 4
 			4 &
			Show a basic understanding of the ability to conduct research, present knowledge in an academic format and apply that research to practice  & 
			Research and explain the use of methodologies used in computing, apply knowledge to practice, and present that knowledge where appropriate in an academic format. \\
			% OUTCOME 6
 			6 &
			Show a basic understanding of methods used to help set goals, manage workloads to meet deadlines and to work collaboratively  & 
			Set goals and manage workloads to meet deadlines using set methodologies and present ideas in a variety of situations with appropriate support. 
		\end{tabular}
	
\end{frame}

\begin{frame}
	\frametitle{Overview}
	
	\centering
		\tiny
		\def\arraystretch{1.5}
		\begin{tabular} { | p{.20\textwidth} | p{.4\textwidth} p{.25\textwidth} |}
		
			%MODULE LEADER
			\textbf{Module Leader} &  
			\moduleLeader &
			\\

			%TEACHING TEAM
			\textbf{Academic Staff} &  
			\academicStaff &
			\\
			
			&&\\
			
			%ASSIGNMENTS
			\textbf{Assignments} &  
			Worksheet Tasks &
			80\% \\
			
			%ASSIGNMENTS
			&  
			Research Journal &
			20\% \\
			
			&&\\
						
			%Hours
			\textbf{Indicative Hours} &  
			Sessions &
			36-hours \\
			
			%Hours
			&
			Directed Reading &
			18-hours \\
			
			%Hours
			&  
			Worksheet Tasks &
			56-hours \\
			
			%Hours
			&
			Research Journals &
			14-hours \\
									
			%Hours
			&
			Self-Directed Computing Study &
			36-hours \\
			
			%Hours
			&
			Self-Directed Portfolio Development &
			40-hours \\
			
		\end{tabular}
		
		\vspace{2em}
		\raggedright
		Each study block represents 600-hours of study. This means that 40 hours of study per week (including contact time) is expected, alongside a further 120-hours of studio practice across the assessment period.

\end{frame}

\begin{frame}
	\frametitle{Additional Resources}
	
	Session Plans \& Materials: \\
	\url{http://learningspace.falmouth.ac.uk/}
	
	\vspace{1.5em}	
	
	Assignment Briefs: \\
	\url{https://github.com/Falmouth-Games-Academy/bsc-assignment-briefs}

	\vspace{1.5em}	
	
	Reading List: \\
	\url{http://resourcelists.falmouth.ac.uk/index.html}

\end{frame}

\end{document}