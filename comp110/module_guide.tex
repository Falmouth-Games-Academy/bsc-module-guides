\documentclass[handout, xcolor={dvipsnames}]{beamer}\usepackage{etoolbox}\newtoggle{printable}\toggletrue{printable}

% Adjust these for the path of the theme and its graphics, relative to this file
\usepackage{../beamerthemeFalmouthGamesAcademy}
\usepackage{multimedia}
\usepackage[nomessages]{fp}
\usepackage{pgf, pgffor}
\usepackage[none]{hyphenat}

\graphicspath{ {../} }

% Additional symbols
\usepackage{wasysym}

\begin{document}

\moduleCode{COMP110}
\moduleName{Principles of Computing}
\moduleLeader{Dr Edward Powley}
\def\academicStaff{Dr Michael Scott (Moderator)}
\credits{20}
\isCompulsory{Compulsory}

\def\introduction{This module is designed to introduce you to the basic principles of computing in the context of digital games. It is designed to complement the other modules, providing a broad foundation on the theories, methods, models, and techniques in computing which will help you to construct computer programs and be able to make use of relevant scholarly sources.}

\def\assignments{
	{Worksheet Tasks, 80\%},
	{Research Journal, 20\%}}

\def\hours{
	{Sessions, 36 hours},
	{Directed Reading, 18 hours},
	{Worksheet Tasks, 56 hours},
	{Research Journal, 14 hours},
	{Self-Directed Study, 36 hours},
	{Self-Directed Studio Practice, 40 hours}}
	
\def\moduleAims{
	{Understand the basic principles, terminology, roles, and software development concept that computing professionals apply within a game development context},
	{Understand how to apply computing theory to practical programming activities},
	{Understand how to conduct basic software development tasks}}
	
\def\moduleLearningOutcomes{1,2,3,4,6}
	
\def\assessmentCriteria{
	{Demonstrate a basic understanding of computing fundamentals. Apply basic knowledge and understanding of the techniques used in software development. Understand the creative value of maker-style and iterative approaches for the generation of innovation.},
	{Show a basic understanding of how to communicate effectively with stakeholders in writing, verbally, and through adherence to coding standards. Annotate software to communicate with others effectively.},
	{Analyse critically the strengths and weaknesses of code and develop an ability to respond to the critical judgements of others.},
	{Research and explain the use of methodologies used in computing, apply knowledge to practice, and present that knowledge where appropriate in an academic format.},
	{},
	{Set goals and manage workloads to meet deadlines using set methodologies and present ideas in a variety of situations with appropriate support.}}

\def\learningSpaceID{1249}

% Make variables available elsewhere in the document
\makeatletter

\let\moduleLeader\@moduleLeader
\let\credits \@credits
\let\moduleCode \@moduleCode

% Calculate total hours of study based on credits
\FPeval{\totalHours}{clip(\credits * 10)}%

% For handling loops inside tables - https://tex.stackexchange.com/questions/165126/how-do-i-use-the-ampersand-inside-a-foreach-or-conditional-or-other-group-e
\newtoks\@tabtoks
\newcommand\addtabtoks[1]{\global\@tabtoks\expandafter{\the\@tabtoks#1}}
\newcommand\eaddtabtoks[1]{\edef\mytmp{#1}\expandafter\addtabtoks\expandafter{\mytmp}}
\newcommand*\resettabtoks{\global\@tabtoks{}}
\newcommand*\printtabtoks{\the\@tabtoks}

\makeatother

\frame{\titlepage} 

\begin{frame}
	\frametitle{Introduction}
	
	\introduction
\end{frame}

\begin{frame}
	\frametitle{Aims}
	
	This module aims to help you:
	
	\begin{itemize}
		\foreach \x in \moduleAims{%
			\item \x	
		}
	\end{itemize}
\end{frame}

\begin{frame}
	\frametitle{Learning Outcomes}	
	
	% ASSESSMENT CRITERIA
	\addtabtoks{\textbf{LO} & \textbf{Learning Outcomes} & \textbf{Assessment Criteria} \\}
	\foreach \x  in \moduleLearningOutcomes{%
		
		\eaddtabtoks{\x}	
		\foreach \y[count=\yi] in \learningOutcomes{%
			\ifnum\x=\yi\relax
				\eaddtabtoks{& \y}	
			\fi
		}
		
		\foreach \y[count=\yi] in \assessmentCriteria{%
			\ifnum\x=\yi\relax
				\eaddtabtoks{& \y}	
			\fi
		}
		\addtabtoks{\\}	
	}
	
	% MAKE TABLE											
	\centering
		\tiny
		\def\arraystretch{1.5}
		\begin{tabular} { | p{.02\textwidth} | p{.45\textwidth} p{.45\textwidth} |}
			\printtabtoks
		\end{tabular}
	\resettabtoks
	
\end{frame}

\begin{frame}
	\frametitle{Overview}
	
	% MODULE LEADER
	\addtabtoks{\textbf{Academic Staff} & \moduleLeader \\}
		
	% OTHER ACADEMIC STAFF
	\foreach \x[count=\xi] in \academicStaff {%
						
		\eaddtabtoks{& \x &}
		\addtabtoks{\\}
	}
		
	\addtabtoks{&&\\}
						
	% ASSIGNMENTS
	\foreach \x[count=\xi] in \assignments {%
		
		\ifnum\xi=1\relax
			\addtabtoks{\textbf{Assignments}}	
		\fi
			
		\foreach \y in \x {%
			
			\eaddtabtoks{& \y}
		}
		\addtabtoks{\\}
	}
		
	\addtabtoks{&&\\}
		
	% HOURS
	\foreach \x[count=\xi] in \hours {%
		
		\ifnum\xi=1\relax
			\addtabtoks{\textbf{Indicative Hours}}	
		\fi
			
		\foreach \y in \x {%
			
			\eaddtabtoks{& \y}
		}
		\addtabtoks{\\}
	}
		
	\addtabtoks{& & \textbf{\totalHours\ hours}}
	\addtabtoks{\\}
			
	% MAKE TABLE											
	\centering
		\tiny
		\def\arraystretch{1.5}
		\begin{tabular} { | p{.20\textwidth} | p{.5\textwidth} p{.1\textwidth} |}
			\printtabtoks
		\end{tabular}
	\resettabtoks
	
	\raggedright
		\vspace{2em}
		Each study block represents 600-hours of study. This means that 40 hours of study per week (including contact time) is expected, alongside a further 120-hours of studio practice across the assessment period.

\end{frame}

\begin{frame}
	\frametitle{Additional Resources}
	
	Session Plans \& Materials: \\
	\url{http://learningspace.falmouth.ac.uk/course/view.php?id=\learningSpaceID}
	
	\vspace{1.5em}	
	
	Assignment Briefs: \\
	\MakeLowercase{\url{http://github.com/Falmouth-Games-Academy/bsc-assignment-briefs/tree/\academicYear/\moduleCode}}
	
	\vspace{1.5em}
	
	Reading List: \\
	\small\MakeLowercase{\url{http://resourcelists.falmouth.ac.uk/modules/\moduleCode}}

\end{frame}

\end{document}