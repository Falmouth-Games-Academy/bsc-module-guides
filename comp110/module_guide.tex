\documentclass[handout, xcolor={dvipsnames}]{beamer}\usepackage{etoolbox}\newtoggle{printable}\toggletrue{printable}

% Adjust these for the path of the theme and its graphics, relative to this file
\usepackage{../beamerthemeFalmouthGamesAcademy}
\usepackage{multimedia}
\graphicspath{ {../} }

% Additional symbols
\usepackage{wasysym}

\begin{document}

\title{COMP110}
\subtitle{Principles of Computing}
\moduleLeader{Dr Ed Powley}
\credits{20}
\isCompulsory{Compulsory}

\frame{\titlepage} 

\begin{frame}
	\frametitle{Introduction}
	This module is designed to introduce you to the basic principles of computing in the context of digital games. It is designed to complement the other modules through providing a broad foundation on the theories, methods, and techniques which will help you to be able to construct computer programs and be able to use relevant scholarly sources.
	
\end{frame}

\begin{frame}
	\frametitle{Aims}
	
	This module aims to help you:
	
	\begin{itemize}
		\item Understand the basic principles, terminology, roles, and software development concept that computing professionals apply within a game development context 
		\item Understand how to apply computing theory to practical programming activities
		\item Understand how to conduct basic software development tasks
	\end{itemize}
\end{frame}

\begin{frame}
	\frametitle{Overview}
	
	Grid goes here.
\end{frame}

\begin{frame}
	\frametitle{Learning Outcomes}
	
	Grid goes here.
\end{frame}

\begin{frame}
	\frametitle{Additional Resources}
	
	Session Plans \& Materials: \\
	\url{http://learningspace.falmouth.ac.uk/}
	
	\vspace{1.5em}	
	
	Assignment Briefs: \\
	\url{https://github.com/Falmouth-Games-Academy/bsc-assignment-briefs}

	\vspace{1.5em}	
	
	Reading List: \\
	\url{http://resourcelists.falmouth.ac.uk/index.html}

\end{frame}

\end{document}