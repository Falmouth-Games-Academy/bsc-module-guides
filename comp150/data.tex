\def\moduleCode{COMP150}
\def\moduleName{Game Development Practice}
\def\credits{20}
\def\isCompulsory{Compulsory}

\def\moduleLeader{Dr Michael Scott}
\def\academicStaff{Dr Edward Powley (Moderator), Brian McDonald (Moderator)}

\def\introduction{This module addresses the foundational principles and processes of computer game development. You will gain a practical understanding of how a playable game comes together according to the agile development philosophy and industry practice. You will also gain a `first principles' understanding of how games are designed with a target market in mind, as well as how creative computing contributes to the process.}

\def\moduleAims{
	{Recall the basic principles, terminology, roles, tools, and pipelines used in the development of digital games},
	{Apply foundational management skills in order to organise and execute a game development project},
	{Communicate and implement version control effectively within a software development team}}
	
\def\moduleLearningOutcomes{1,2,3,4,5,6}
	
\def\assessmentCriteria{
	{Apply basic knowledge and understanding of the professional techniques used to create digital games and employ elementary principles of game development to devise a simple game concept using Agile and iterative methods.},
	{Organise your ideas and material to communicate clearly with others; have a working knowledge of Agile methods.},
	{Identify and appraise the main strengths and weakness of your working methods and solutions.},
	{Research uses of Agile methods and supports within the context of game development.},
	{Show a basic understanding of the commercial and enterprise context of the games industry and the professional qualities needed for decision-making within that context.},
	{Deliver a collective game concept on time and to brief, responding appropriately to problems and changes in direction. Choose appropriate means to convey your development ideas.}}

\def\assignments{
	{Agile Essay, 30\%},
	{Pre-Production Tasks, 40\%},
	{Game Design Pitches, 10\%},
	{CPD Tasks, 20\%}}

\def\hours{
	{Sessions, 36 hours},
	{Supervised Studio Practice, 42 hours},
	{Directed Reading, 12 hours},
	{Agile Essay, 21 hours},
	{Pre-Production Tasks, 28 hours},
	{Game Design \& Pitch Preparation, 7 hours},
	{CPD Tasks, 14 hours},
	{Self-Directed Studio Practice, 40 hours}}
	
\def\learningSpaceID{2998}