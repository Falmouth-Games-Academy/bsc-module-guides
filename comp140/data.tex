\moduleCode{COMP140}
\moduleName{Creative Computing: Codecraft}
\credits{20}
\isCompulsory{Compulsory}

\moduleLeader{Brian McDonald}
\def\academicStaff{Alcywn Parker (Physical Computing Specialist), Martin Cooke (Moderator)}

\def\introduction{This module builds upon Creative Computing: Tinkering, allowing you to further develop confidence with object-orientated programming and the creative approach to computing in the games development context. You will take code in multiple contexts, and learn ways and methods for bringing these together in synthesis in order to build more interesting and complex systems. Part of this will involve ‘hacking’ together different sets of open-source code, hardware, and web services together; all the while considering issues such as intellectual property law and licensing.}

\def\moduleAims{
	{Understand the role of the computing professional in the games industry},
	{Understand how to organise, repurpose, and augment code from multiple sources to build a unified solution},
	{Understand how to generate innovation at a basic level}}
	
\def\moduleLearningOutcomes{1,3,5,6}
	
\def\assessmentCriteria{
	{To modify and repurpose existing code from multiple sources and apply the basic principles of software engineering to solve problems.},
	{},
	{Evaluate your working practice showing that you understand the analytical approach required to learn from your practical work.},
	{},
	{To creatively repurpose existing code from multiple sources towards a unified solution and use a combination of sources to generate ideas and new solutions.},
	{Meet deadlines by planning available time effectively and show an understanding of how to plan and manage time.}}

\def\assignments{
	{Code Combination I --- API Tasks, 30\%},
	{Code Combination II --- Individual Game \& Controller, 70\%}}

\def\hours{
	{Sessions, 36 hours},
	{Directed Reading, 18 hours},
	{API Tasks, 24 hours},
	{Individual Game \& Controller, 46 hours},
	{Self-Directed Study, 36 hours},
	{Self-Directed Studio Practice, 40 hours}}
	
\def\learningSpaceID{1252}