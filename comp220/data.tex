\def\moduleCode{COMP220}
\def\moduleName{Graphics \& Simulation}
\def\credits{20}
\def\isCompulsory{Compulsory}

\def\moduleLeader{Brian McDonald}
\def\academicStaff{Dr Edward Powley (Moderator)}

\def\introduction{On this module you will develop your understanding of 3D graphics rendering and physics simulation used in modern computer games. Using Modern Graphics APIs, you will develop your coding skills in the context of graphics technologies and pipelines. You will also engage practically and creatively to develop physics simulation and rendering pipelines in order to support your individual or group game concept.}

\def\moduleAims{
	{Gain an understanding and knowledge of graphics and simulation technology},
	{Build an understanding of rendering and physics pipelines},
	{Gain experience of how to creatively leverage the capabilities of graphics and simulation technologies}}
	
\def\moduleLearningOutcomes{1,2,3,4,6}
	
\def\assessmentCriteria{
	{Change the way that a graphics engine behaves and demonstrate an understanding of graphics rendering engines.},
	{Show a basic understanding of how to communicate effectively with stakeholders in writing, verbally, and through adherence to coding standards. Annotate software to communicate with others effectively.},
	{Reflect critically on the behaviour change intended and its visual structure and explain the rationale for working method and graphics-based solution.},
	{Understand and apply knowledge of rendering pipelines used to produce changes in graphics engine behaviour.},
	{},
	{Show understanding of how to plan and organise time to meet deadlines and fulfil a brief.}}

\def\assignments{
	{Portfolio of Game Engine Components, 90\%},
	{Research Journal, 10\%}}

\def\hours{
	{Sessions, 36 hours},
	{Directed Reading, 18 hours},
	{Portfolio of Game Engine Components, 55 hours},
	{Integration into Collaborative Game, 20 hours},
	{Research Journal, 7 hours},
	{Self-Directed Study, 24 hours},
	{Self-Directed Studio Practice, 40 hours}}
	
\def\learningSpaceID{1255}