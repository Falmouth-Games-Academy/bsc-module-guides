\def\moduleCode{COMP310}
\def\moduleName{Legacy Game Systems}
\def\credits{20}
\def\isCompulsory{Compulsory}

\def\moduleLeader{Dr Edward Powley}
\def\academicStaff{Brian McDonald (Moderator)}

\def\introduction{On this module you build on your experience and knowledge of programming by engaging with the underlying computer technology in greater depth through an exploration of legacy game systems. You will learn the importance of disciplined programming practice while using low-level languages to create a simple game prototype. You'll demonstrate this using an emulator for a legacy game platform.}

\def\moduleAims{
	{Acquire knowledge and understanding of professional software engineering in the context of legacy technology.},
	{Understand low-level computing principles and processor architectures.},
	{Apply low-level computing knowledge to the development of games and game-related software.}}
	
\def\moduleLearningOutcomes{1,2,3,4,6}
	
\def\assessmentCriteria{
	{Understand the fundamental use of cross-platform development tools and how constraints vary between different platforms.},
	{Understand the importance of legibility at all levels of software development.},
	{Analyse critically the strengths and weaknesses of assembly code.},
	{Apply basic research methodologies to understand historical developments in legacy platform capabilities and evolution.},
	{},
	{Meet deadlines by planning available time effectively.}}

\def\assignments{
	{Constrained Development Task, 80\%},
	{Research Journal, 20\%}}

\def\hours{
	{Sessions, 27 hours},
	{Directed Reading, 18 hours},
	{Constrained Development Task, 56 hours},
	{Integration into Collaborative Game, 20 hours},
	{Research Journal, 15 hours},
	{Self-Directed Study, 24 hours},
	{Self-Directed Studio Practice, 40 hours}}
	
\def\learningSpaceID{1506}