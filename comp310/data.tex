\def\moduleCode{COMP310}
\def\moduleName{Legacy Game Systems}
\def\credits{20}
\def\isCompulsory{Compulsory}

\def\moduleLeader{Dr Edward Powley}
\def\academicStaff{Brian McDonald (Moderator)}

\def\introduction{On this module you will engage with interface technologies which are changing the way that we play games. You will undertake a series of practical and creative engagements with emergent technologies, such as augmented and virtual reality devices, working iteratively to produce an innovative solution. You may tie this work into either your individual or collaborative game development project.}

\def\moduleAims{
	{Acquire knowledge and understanding of professional software engineering in the context of legacy games},
	{Understand the role of low-level computing principles and processor architectures in the execution of game software.},
	{Apply low-level computing knowledge to the development of games and game-related software.}}
	
\def\moduleLearningOutcomes{1,2,3,4,6}
	
\def\assessmentCriteria{
	{Understand the fundamental use of cross-platform development tools and how constraints vary between different platforms.},
	{Understand the importance of legibility at all levels of software development.},
	{Analyse critically the strengths and weaknesses of assembly code.},
	{Apply basic research methodologies to understand historical developments in legacy platform capabilities and evolution.},
	{},
	{Meet deadlines by planning available time effectively.}}

\def\assignments{
	{Constrained Development Task, 80\%},
	{Research Journal, 20\%}}

\def\hours{
	{Sessions, 27 hours},
	{Directed Reading, 18 hours},
	{Constrained Development Task, 56 hours},
	{Integration into Collaborative Game, 20 hours},
	{Research Journal, 15 hours},
	{Self-Directed Study, 24 hours},
	{Self-Directed Studio Practice, 40 hours}}
	
\def\learningSpaceID{???}