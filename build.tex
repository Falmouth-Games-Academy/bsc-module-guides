\documentclass[10pt, a5paper, twoside, openright, titlepage]{memoir}\usepackage{etoolbox}\newtoggle{printable}\toggletrue{printable}\newtoggle{isBeamer}\togglefalse{isBeamer}
\usepackage[top=100pt, bindingoffset=18mm, heightrounded]{geometry}
\usepackage{beamerthemeFalmouthGamesAcademy}

\usepackage{url}
\makeatletter
\g@addto@macro{\UrlBreaks}{\UrlOrds}
\makeatother

\makeatletter
\newcommand\thickhrulefill{\leavevmode \leaders \hrule height 1ex \hfill \kern \z@}
\setlength\midchapskip{10pt}

\makechapterstyle{VZ14}{
\renewcommand\chapternamenum{}
\renewcommand\printchaptername{}
%\renewcommand\chapnamefont{\Large\scshape}
\renewcommand\printchapternum{%
\chapnamefont\null\thickhrulefill\quad
\@chapapp\space\thechapter\quad\thickhrulefill}
\renewcommand\printchapternonum{%
\par\thickhrulefill\par\vskip\midchapskip
\hrule\vskip\midchapskip
}
\renewcommand\chaptitlefont{\Huge\centering}
\renewcommand\afterchapternum{%
\par\nobreak\vskip\midchapskip\hrule\vskip\midchapskip}
\renewcommand\afterchaptertitle{%
\par\vskip\midchapskip\hrule\nobreak\vskip\afterchapskip}
}
\makeatother
\chapterstyle{VZ14}

%% BEGIN TITLE

\author{Falmouth University Games Academy}
\title{BSc(Hons) Computing for Games}
\date{2017-18}

%\renewcommand{\chaptername}{Section}
%\renewcommand\cftchapteraftersnumb{\normalfont}
\textwidth=9cm

\setsecnumdepth{chapter}
\setlength{\parindent}{0em}

\renewcommand{\chaptername}{Section}

%\pagestyle{ruled} % activate changes

\addtopsmarks{headings}{}{
   \createmark{chapter}{left}{nonumber}{}{}
}
\pagestyle{headings} % activate changes

%\renewcommand\cftbeforechapterskip{10pt plus 1pt}

\usepackage{pdfcolmk}
\usepackage[T1]{fontenc}
\usepackage{lmodern}
\usepackage{pifont}

\usepackage{avant}
\renewcommand{\familydefault}{\sfdefault}

%% select a (FontSite) font by its font family ID
%\def\drop{1}
\newcommand*{\FSfont}[1]{\fontencoding{T1}\fontfamily{#1}\selectfont}

\newcommand*{\plogo}{\fbox{$\mathcal{PL}$}}

\newcommand*{\titleRF}{
%\thispagestyle{titlepage}
\begingroup% Robert Frost, T&H p 149
%\FSfont{2bp} % FontSite Bergamo (Bembo)
\drop = 0.2\textheight
\centering
{\vspace{1cm}
\huge BSc(Hons) Computing for Games}\\[\baselineskip]
{\Huge Student Handbook}\\[\baselineskip]
{\large 2017-18}\\[0.5\drop]
\vfill
{\includegraphics[height=4em]{GamesLogoAt}	
\includegraphics[height=2em]{FalmouthLogo}}\\[0.5\baselineskip]
\endgroup}

\newlength{\drop}% for my convenience

%%% BEGIN DOCUMENT

\begin{document}

%\begin{titlingpage}
%\let\cleardoublepage\newpage
\thispagestyle{empty}
\titleRF
\pagebreak
%\end{titlingpage}

\thispagestyle{empty}
\null\vfill

\begin{small}
\begin{flushleft}
\textit{BSc(Hons) Computing for Games --- Student Handbook}

\copyright  2017 Falmouth University. All rights reserved.

\bigskip

DOI: \url{10.13140/2.1.3680.9281}

\vspace{3em}

\includegraphics[height=2em]{ccbyncnd4-88x31}	

\footnotesize This work may be distributed and/or modified under the conditions of the Creative Commons BY-NC-SA 4.0 license. The latest version is available in:

\smallskip

%\begin{scriptsize}
\url{https://github.com/Falmouth-Games-Academy/bsc-module-guides}
%\end{scriptsize}

\end{flushleft}
\end{small}

%\end{changemargin}

%\let\cleardoublepage\clearpage

\frontmatter

\begin{small}
	\addtolength{\voffset}{-3.2cm}
	\addtolength{\textheight}{3.2cm}
	\tableofcontents
\end{small}

\newpage

\section*{External Examiner's Comments}
\addcontentsline{toc}{chapter}{Forward}

Now in its second year, the \textit{BSc Computing for Games} course is holding up well. The team have made a number of small improvements to help the incoming students while the pioneers continue. The second-year modules are quite varied in scope and interesting in content. I am amazed students get the to design and make a MUD here! I love this! What a great idea for teaching distributed systems.\\

The relevancy of computing is on an upwards trajectory. All the while, the currency of the Falmouth award exceeds that of awards at similar institutions. \\

Employability is written through the degree like Falmouth in a stick of rock. This is a major selling point, and what makes Falmouth's offering so distinctive. Students are treated as if they are already working in a studio. This approach is quite demanding on students, but they are all the better for it. Graduates will have been working in a system sufficiently similar to that of the games industry that it is as if they have years of employment to their credit already. \\

There is some excellent teaching at Falmouth. Assessments are well designed and the marking is fair. It is also of an astonishingly high quality. I hope the students appreciate this, because it really is an order of magnitude better than what I typically see elsewhere.

\bigskip

\begin{flushright}

--- Professor Richard Bartle, \\
Essex University

\bigskip

Co-creator of \textit{MUD1}
\end{flushright}

%\clearpage
\mainmatter

\chapter{Study Block I}
\newpage

\def\moduleList{
	comp110, 
	comp120, 
	comp150,
	comp210,
	comp220,
	comp230,
	comp310,
	comp320,
	comp330}
	 
\foreach \x in \moduleList {

	\def\moduleCode{COMP220}
\def\moduleName{Graphics \& Simulation}
\def\credits{20}
\def\isCompulsory{Compulsory}

\def\moduleLeader{Brian McDonald}
\def\academicStaff{Dr Edward Powley (Moderator)}

\def\introduction{On this module you will develop your understanding of 3D graphics rendering and physics simulation used in modern computer games. Using Modern Graphics APIs, you will develop your coding skills in the context of graphics technologies and pipelines. You will also engage practically and creatively to develop physics simulation and rendering pipelines in order to support your individual or group game concept.}

\def\moduleAims{
	{Gain an understanding and knowledge of graphics and simulation technology},
	{Build an understanding of rendering and physics pipelines},
	{Gain experience of how to creatively leverage the capabilities of graphics and simulation technologies}}
	
\def\moduleLearningOutcomes{1,2,3,4,6}
	
\def\assessmentCriteria{
	{Change the way that a graphics engine behaves and demonstrate an understanding of graphics rendering engines.},
	{Show a basic understanding of how to communicate effectively with stakeholders in writing, verbally, and through adherence to coding standards. Annotate software to communicate with others effectively.},
	{Reflect critically on the behaviour change intended and its visual structure and explain the rationale for working method and graphics-based solution.},
	{Understand and apply knowledge of rendering pipelines used to produce changes in graphics engine behaviour.},
	{Show understanding of how to plan and organise time to meet deadlines and fulfil a brief.}}

\def\assignments{
	{Portfolio of Game Engine Components, 90\%},
	{Research Journal, 10\%}}

\def\hours{
	{Sessions, 36 hours},
	{Directed Reading, 18 hours},
	{Portfolio of Game Engine Components, 55 hours},
	{Integration into Collaborative Game, 20 hours},
	{Research Journal, 7 hours},
	{Self-Directed Study, 24 hours},
	{Self-Directed Studio Practice, 40 hours}}
	
\def\learningSpaceID{1255}
	\FPeval{\totalHours}{clip(\credits * 10)}%

\includegraphics[height=3em]{GamesLogoAt}	

\section{\moduleName}

	% MODULE CODE
	\addtabtoks{\textbf{Module Code} & \moduleCode&\\}
	
	\addtabtoks{&&\\}
	
	\addtabtoks{\textbf{Module Credits} & \credits&\\}
	
	\addtabtoks{&&\\}
	
	\addtabtoks{\textbf{Status} & \isCompulsory&\\}
	
	\addtabtoks{&&\\}

	% MODULE LEADER
	\addtabtoks{\textbf{Module Leader} & \moduleLeader & \\}
		
	% OTHER ACADEMIC STAFF
	\foreach \x[count=\xi] in \academicStaff {%
						
		\eaddtabtoks{& \x &}
		\addtabtoks{\\}
	}
		
	\addtabtoks{&&\\}
						
	% ASSIGNMENTS
	\foreach \x[count=\xi] in \assignments {%
		
		\ifnum\xi=1\relax
			\addtabtoks{\textbf{Assignments}}	
		\fi
			
		\foreach \y in \x {%
			
			\eaddtabtoks{& \y}
		}
		\addtabtoks{\\}
	}
		
	\addtabtoks{&&\\}
		
	% HOURS
	\foreach \x[count=\xi] in \hours {%
		
		\ifnum\xi=1\relax
			\addtabtoks{\textbf{Indicative Hours}}	
		\fi
			
		\foreach \y in \x {%
			
			\eaddtabtoks{& \y}
		}
		\addtabtoks{\\}
	}
		
	\addtabtoks{& & \textbf{\totalHours\ hours}}
	\addtabtoks{\\}
			
	% MAKE TABLE											
	\begin{center}
		\footnotesize
		\def\arraystretch{1.1}
		\begin{tabular} { | p{.23\textwidth} | p{.63\textwidth} p{.14\textwidth} |}
			\printtabtoks
		\end{tabular}
	\end{center}
	\resettabtoks
	
	\newpage
	
	\subsection{Introduction}
	
	\introduction
	
	\subsection{Aims}
	
	This module aims to help you:
	
	\begin{itemize}
		\foreach \x in \moduleAims{%
			\item \x	
		}
	\end{itemize}

	\subsection{Resource List}
	\MakeLowercase{\protect\url{http://resourcelists.falmouth.ac.uk/modules/\moduleCode}}

	\subsection{Learning Space}
	\MakeLowercase{\protect\url{http://learningspace.falmouth.ac.uk/course/view.php?id=\learningSpaceID}}
	
	\newpage
		
	% ASSESSMENT CRITERIA
	\addtabtoks{\textbf{LO} & \textbf{Learning Outcomes} & \textbf{Assessment Criteria} \\}
	\foreach \x  in \moduleLearningOutcomes{%
		
		\eaddtabtoks{\x}	
		\foreach \y[count=\yi] in \learningOutcomes{%
			\ifnum\x=\yi\relax
				\eaddtabtoks{& \y}	
			\fi
		}
		
		\foreach \y[count=\yi] in \assessmentCriteria{%
			\ifnum\x=\yi\relax
				\eaddtabtoks{& \y}	
			\fi
		}
		\addtabtoks{\\}	
	}
	
	% MAKE TABLE											
	\begin{center}
		\footnotesize
		\def\arraystretch{1.25}
		\hspace*{-0.1cm}\begin{tabular} { | P{.03\textwidth} | p{.47\textwidth} p{.47\textwidth} |}
			\printtabtoks
		\end{tabular}
	\resettabtoks
	\end{center}
	

	\newpage
}
	
\chapter{Study Block II}
\newpage
	
\def\moduleList{	   
	comp140, 
	comp130,
	comp250,
	comp260,
	comp240,
	comp350,
	comp360,
	comp340}
	
\foreach \x in \moduleList {

	\def\moduleCode{COMP220}
\def\moduleName{Graphics \& Simulation}
\def\credits{20}
\def\isCompulsory{Compulsory}

\def\moduleLeader{Brian McDonald}
\def\academicStaff{Dr Edward Powley (Moderator)}

\def\introduction{On this module you will develop your understanding of 3D graphics rendering and physics simulation used in modern computer games. Using Modern Graphics APIs, you will develop your coding skills in the context of graphics technologies and pipelines. You will also engage practically and creatively to develop physics simulation and rendering pipelines in order to support your individual or group game concept.}

\def\moduleAims{
	{Gain an understanding and knowledge of graphics and simulation technology},
	{Build an understanding of rendering and physics pipelines},
	{Gain experience of how to creatively leverage the capabilities of graphics and simulation technologies}}
	
\def\moduleLearningOutcomes{1,2,3,4,6}
	
\def\assessmentCriteria{
	{Change the way that a graphics engine behaves and demonstrate an understanding of graphics rendering engines.},
	{Show a basic understanding of how to communicate effectively with stakeholders in writing, verbally, and through adherence to coding standards. Annotate software to communicate with others effectively.},
	{Reflect critically on the behaviour change intended and its visual structure and explain the rationale for working method and graphics-based solution.},
	{Understand and apply knowledge of rendering pipelines used to produce changes in graphics engine behaviour.},
	{Show understanding of how to plan and organise time to meet deadlines and fulfil a brief.}}

\def\assignments{
	{Portfolio of Game Engine Components, 90\%},
	{Research Journal, 10\%}}

\def\hours{
	{Sessions, 36 hours},
	{Directed Reading, 18 hours},
	{Portfolio of Game Engine Components, 55 hours},
	{Integration into Collaborative Game, 20 hours},
	{Research Journal, 7 hours},
	{Self-Directed Study, 24 hours},
	{Self-Directed Studio Practice, 40 hours}}
	
\def\learningSpaceID{1255}
	\FPeval{\totalHours}{clip(\credits * 10)}%

\includegraphics[height=3em]{GamesLogoAt}	

\section{\moduleName}

	% MODULE CODE
	\addtabtoks{\textbf{Module Code} & \moduleCode&\\}
	
	\addtabtoks{&&\\}
	
	\addtabtoks{\textbf{Module Credits} & \credits&\\}
	
	\addtabtoks{&&\\}
	
	\addtabtoks{\textbf{Status} & \isCompulsory&\\}
	
	\addtabtoks{&&\\}

	% MODULE LEADER
	\addtabtoks{\textbf{Module Leader} & \moduleLeader & \\}
		
	% OTHER ACADEMIC STAFF
	\foreach \x[count=\xi] in \academicStaff {%
						
		\eaddtabtoks{& \x &}
		\addtabtoks{\\}
	}
		
	\addtabtoks{&&\\}
						
	% ASSIGNMENTS
	\foreach \x[count=\xi] in \assignments {%
		
		\ifnum\xi=1\relax
			\addtabtoks{\textbf{Assignments}}	
		\fi
			
		\foreach \y in \x {%
			
			\eaddtabtoks{& \y}
		}
		\addtabtoks{\\}
	}
		
	\addtabtoks{&&\\}
		
	% HOURS
	\foreach \x[count=\xi] in \hours {%
		
		\ifnum\xi=1\relax
			\addtabtoks{\textbf{Indicative Hours}}	
		\fi
			
		\foreach \y in \x {%
			
			\eaddtabtoks{& \y}
		}
		\addtabtoks{\\}
	}
		
	\addtabtoks{& & \textbf{\totalHours\ hours}}
	\addtabtoks{\\}
			
	% MAKE TABLE											
	\begin{center}
		\footnotesize
		\def\arraystretch{1.1}
		\begin{tabular} { | p{.23\textwidth} | p{.63\textwidth} p{.14\textwidth} |}
			\printtabtoks
		\end{tabular}
	\end{center}
	\resettabtoks
	
	\newpage
	
	\subsection{Introduction}
	
	\introduction
	
	\subsection{Aims}
	
	This module aims to help you:
	
	\begin{itemize}
		\foreach \x in \moduleAims{%
			\item \x	
		}
	\end{itemize}

	\subsection{Resource List}
	\MakeLowercase{\protect\url{http://resourcelists.falmouth.ac.uk/modules/\moduleCode}}

	\subsection{Learning Space}
	\MakeLowercase{\protect\url{http://learningspace.falmouth.ac.uk/course/view.php?id=\learningSpaceID}}
	
	\newpage
		
	% ASSESSMENT CRITERIA
	\addtabtoks{\textbf{LO} & \textbf{Learning Outcomes} & \textbf{Assessment Criteria} \\}
	\foreach \x  in \moduleLearningOutcomes{%
		
		\eaddtabtoks{\x}	
		\foreach \y[count=\yi] in \learningOutcomes{%
			\ifnum\x=\yi\relax
				\eaddtabtoks{& \y}	
			\fi
		}
		
		\foreach \y[count=\yi] in \assessmentCriteria{%
			\ifnum\x=\yi\relax
				\eaddtabtoks{& \y}	
			\fi
		}
		\addtabtoks{\\}	
	}
	
	% MAKE TABLE											
	\begin{center}
		\footnotesize
		\def\arraystretch{1.25}
		\hspace*{-0.1cm}\begin{tabular} { | P{.03\textwidth} | p{.47\textwidth} p{.47\textwidth} |}
			\printtabtoks
		\end{tabular}
	\resettabtoks
	\end{center}
	

	\newpage
}

\end{document}